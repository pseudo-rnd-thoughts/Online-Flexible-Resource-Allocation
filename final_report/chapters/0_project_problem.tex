\chapter{Project problem}\label{ch:project-problem}
Cloud computing is a rapidly growing service with competition from Google, Amazon, Microsoft and more that aims to
allow users to run computer programs that are too large, difficult or time consuming for users to run locally.
These services provide the computational resources, e.g.\ cpu cores, RAM, hard drive space, bandwidth, etc
to be able to run such programs. However, as these resources are limited, bottlenecks can occur when
numerous users all require large amounts of these resources, limiting the number of tasks
\footnote{Tasks, Programs and Jobs will be used interchangeable to refer to the same idea of a computer programs that
has a fixed amount of resources required to compute.} that can be run on the cloud servers simultaneously.

For Google Cloud Services (GCP), Microsoft Azure or Amazon Web Services, their cloud computing facilities contain huge
server nodes limiting the probability of such bottlenecking occurring and if such an event does occur. Users also have
a range of data centres across the global to use instead if a single data centre becomes overloaded.
Therefore this work considers still a developing paradigm~\citep{mobile_edge_survey} called Edge/Mobile cloud computing.
Edge cloud computing is believed to have a wide range of application where traditional cloud computing would be
impractical. This could be due to tasks being highly delay-sensitive, intermittent internet connectivity
or high operational security that prevent or limit the effectiveness of traditional cloud computing.

Currently disaster response~\cite{mobile_edge_disaster}, smart cities~\cite{smart_disaster_management} and
internet-of-things~\citep{mobile_edge_IoT} (IoT) are all area that utilise edge cloud computing due to its ability
to process computationally small tasks locally with low latency. For example, in smart cities, this
allows for smart intersection systems using of road-side sensors or smart traffic lights based
on cameras to minimise the waiting times~\citep{smart_cities_traffic_lights}. Or for the police to analysis
CCTV footage to spot suspicious behaviour or to track people between cameras~\citep{Sreenu2019}. In the case
of disaster response, maps can be produced using data from autonomous vehicles sensors that can then be used in the
search for potential victims and support responders~\citep{smart_disaster_management}.

However the problem of bottlenecking is of particular relevant in edge cloud computing, as instead of large server farms
that can be geographically distant from the users. Edge cloud computing server are significantly smaller,
being just high powered desktop computers and single server nodes. This results in greater demand on server resources,
meaning that efficient allocation of resources is extremely important. Because of this, resource allocation in edge
cloud computing is an important and interesting research area with edge cloud computing.

However it is believed that there is a shortcoming of existing work in resource allocation within
edge cloud computing~\citep{vaji_infocom, Bi2019} due to the nature of task resource requirements. Traditional, to
compute a task, several types of resource are required to be allocated for the task including communication bandwidth,
computational power and data storage resources~\citep{vaji_infocom}.
However this locks away these resources, preventing any other users have using these resources till the task is
finished with them. But this has the disadvantage bottlenecking occurring due to the inflexible
nature of resources that cannot allow rebalancing of resources between tasks. The reason for the continued use of fixed
resource allocation methods is that in traditional cloud computing such bottlenecking is rare and it allows for
simpler pricing mechanism as the price can be proportional to the quantity of resources required.

Previous work by this author~\citep{FlexibleResourceAllocation} proposed a novel resource allocation method to allow
for significantly more flexibility in resource allocation with the aims of reducing possible bottlenecking. This was
done by requesting that users provide a total resource usage over a tasks lifetime instead of the required resource
usage. With this change of how much of each resource to be allocated could be decided by the server instead of the user.
This is possible as the time taken for an operation is complete is generally proportional to the resources provided
for the operation. An example for this is downloading an image, the time taken is proportional to the bandwidth
allocated. This sort of flexibility is similarly true for computing a task or sending results back to the user as well.
Therefore using a deadline, provided by the user, it is possible to reallocate resources around tasks to reduce the
overall strain on certain resources while still finishing the task with its deadline. Therefore Using this alternative
resource allocation procedure, bottlenecks can be prevent through proper balancing of resources that in turn will allow
more tasks to run simultaneously and to reduce the price for user to run a task.

But in previous work~\citep{FlexibleResourceAllocation}, this resource allocation method was only considered in a static
or one-shot approach where all tasks were presented at the first time step. At which point tasks would be auctioned
and resource allocated. As a result while tasks would be processed in batches such that servers would bid on all tasks
submitted every 5 minutes or so. Therefore previous work could also not dynamically change the resources allocated
between batches making it impractical to be used commercially, this work aims to address these problems in previous work.

These problems are addressed by introducing time into the optimisation problem (outlined in
section~\ref{sec:optimisation-problem}) but due to this addition, all previous mechanism proposed
in~\cite{FlexibleResourceAllocation} are incompatible with the new optimisation problem. Therefore using a standard
auction mechanism, this project investigates different methods of learning how to bid on tasks based on their resource
requirements and to efficiently allocate resources to tasks by a server.

This report is set out is following chapters. Chapter~\ref{ch:proposed-solution-to-problem} proposed a solution
to the project outline in this chapter with chapter~\ref{ch:justification-of-the-solution} justifying why this
approach as taken over alternative. Chapter~\ref{ch:background-literature} investigates the previous research
that this project builds upon within both resource allocation in cloud computing and reinforcement learning methods.
The proposed solution is then implemented in chapter~\ref{ch:implementation-of-the-solution} with testing and
evaluation in chapters~\ref{ch:testing-of-the-implementation} and~\ref{ch:evaluation-of-the-implementation} respectively.
