%% ----------------------------------------------------------------
%% Project.tex
%% ---------------------------------------------------------------- 
\documentclass[sotoncolour]{uosproject}     % Use the Project Style with custom link colour
\usepackage[round]{natbib}     % Use Natbib style for the refs.
\usepackage{bibentry}          % Use bibentry for prepublished works
\usepackage{pdfpages}          % Enables PDF to be inserted into the project
\nobibliography*               % Use bibentry for prepublished works
\usepackage{attrib}            % Use the attrib package for quotations
\usepackage{}
\hypersetup{colorlinks=true}   % Set to false for black/white printing
%% ----------------------------------------------------------------
%% Definitions.tex
%% ---------------------------------------------------------------- 
\newcommand{\BibTeX}{{\rm B\kern-.05em{\sc i\kern-.025em b}\kern-.08em T\kern-.1667em\lower.7ex\hbox{E}\kern-.125emX}}

%% People
\newcounter{address}
\setcounter{address}{1}
\renewcommand{\theaddress}{\textsuperscript{\fnsymbol{address}}}
\newcommand{\address}[1]{\refstepcounter{address}\theaddress#1\\}
\newcommand{\Name}[3]{\texorpdfstring{\href{mailto:#3}{#2}#1}{#2}\xspace}
\newcommand{\SteveRGunn}[1]{\Name{#1}{Steve R. Gunn}{S.R.Gunn@ecs.soton.ac.uk}}

%% Dingbats
\newcommand{\tick}{\ding{51}}
\newcommand{\cross}{\ding{55}}

%% Calculus
\newcommand{\pd}[2]{\ensuremath{\frac{\partial #1}{\partial #2}}\xspace}
\newcommand{\fd}[2]{\ensuremath{\frac{d #1}{d #2}}\xspace}
\newcommand{\dint}{\ensuremath{\int\!\!\!\int}\xspace}
\newcommand{\tint}{\ensuremath{\int\!\!\!\int\!\!\!\int}\xspace}

%% Math Sets
\newcommand{\Q}[1]{\ensuremath{\mathbb{#1}}\xspace}
\newcommand{\R}{\Q{R}}

%% Matrix, Vector
\newcommand{\V}[1]{\ensuremath{\boldsymbol{#1}}\xspace}
\newcommand{\M}[1]{\ensuremath{\boldsymbol{#1}}\xspace}
\newcommand{\0}{\V{0}}
\newcommand{\1}{\V{1}}
\newcommand{\I}{\M{I}}

%% Math Functions
\newcommand{\F}[1]{\ensuremath{\mathrm{#1}}\xspace}
\newcommand{\sgn}{\F{sgn}}
\newcommand{\tr}{\F{trace}}
\newcommand{\diag}{\F{diag}}

%% Math Names
\newcommand{\N}[1]{\ensuremath{\mathit{#1}}\xspace}

%% Data
\newcommand{\mc}[1]{\ensuremath{\mathcal{#1}}\xspace}
\newcommand{\Hyp}{\mc{H}}
\newcommand{\D}{\mc{D}}

%% Kernel
\newcommand{\K}{\M{K}}
\newcommand{\eins}{\texorpdfstring{\ensuremath{\epsilon}}{\textepsilon}-insensitive\xspace}
\newcommand{\e}{\ensuremath{\epsilon}\xspace}
\newcommand{\Bxi}{\ensuremath{\boldsymbol{\xi}}\xspace}
\newcommand{\Kanova}{\ensuremath{\mathit{K_{ANOVA}}}\xspace}
\newcommand{\Kspline}{\ensuremath{\mathit{K_{spline}}}\xspace}

%% Bayesian
\newcommand{\MP}{\ensuremath{\mathit{{\scriptscriptstyle \hspace{-1.5pt}M\hspace{-1.5pt}P}}}\xspace}
\newcommand{\ML}{\ensuremath{\mathit{{\scriptscriptstyle \hspace{-1.5pt}M\hspace{-1.5pt}L}}}\xspace}
\newcommand{\Qw}{\ensuremath{Q_{\w}(\w)}\xspace}
\newcommand{\Qa}{\ensuremath{Q_{\Ba}(\Ba)}\xspace}
\newcommand{\Qb}{\ensuremath{Q_{\beta}(\beta)}\xspace}
\newcommand{\wMPab}{\ensuremath{\w_{\MP|\bar {\Ba},\bar \beta}}\xspace}
\newcommand{\wMP}{\ensuremath{\w_{\MP}}\xspace}
\newcommand{\yMP}{\ensuremath{y_{\MP}}\xspace}
\newcommand{\BaMP}{\ensuremath{\Ba_{\hspace{1pt}\MP}}\xspace}
\newcommand{\aMP}{\ensuremath{\alpha_{\hspace{1pt}\MP}}\xspace}
\newcommand{\bMP}{\ensuremath{\beta_{\hspace{1pt}\MP}}\xspace}
\newcommand{\Sab}{\ensuremath{\M{\Sigma}_{\bar \Ba,\bar \beta}}\xspace}
\newcommand{\Ba}{\ensuremath{\boldsymbol{\alpha}}\xspace}
\newcommand{\Bb}{\ensuremath{\boldsymbol{\beta}}\xspace}
\newcommand{\Bm}{\ensuremath{\boldsymbol{\mu}}\xspace}
\newcommand{\BL}{\ensuremath{\boldsymbol{\Lambda}}\xspace}
\newcommand{\BPhi}{\ensuremath{\boldsymbol{\Phi}}\xspace}
\newcommand{\SMP}{\ensuremath{\M{\Sigma}_{\MP}}\xspace}

\newcommand{\Pa}{\ensuremath{P(\alpha|\mathcal{H})}\xspace}
\newcommand{\Pb}{\ensuremath{P(\beta|\mathcal{H})}\xspace}
\newcommand{\Pab}{\ensuremath{P(\alpha,\beta|\mathcal{H})}\xspace}
\newcommand{\Pw}{\ensuremath{P(\w|\mathcal{H})}\xspace}
\newcommand{\PD}{\ensuremath{P(\D|\mathcal{H})}\xspace}
\newcommand{\PwIa}{\ensuremath{P(\w|\alpha,\mathcal{H})}\xspace}
\newcommand{\PDIwb}{\ensuremath{P(\D|\w,\beta,\mathcal{H})}\xspace}
\newcommand{\PDwab}{\ensuremath{P(\D,\w,\alpha,\beta|\mathcal{H})}\xspace}
\newcommand{\PDIw}{\ensuremath{P(\D|\w,\mathcal{H})}\xspace}
\newcommand{\PwID}{\ensuremath{P(\w|\D,\mathcal{H})}\xspace}
\newcommand{\PwabID}{\ensuremath{P(\w,\alpha,\beta|\D,\mathcal{H})}\xspace}

\newcommand{\PanH}{\ensuremath{P(\alpha)}\xspace}
\newcommand{\PbnH}{\ensuremath{P(\beta)}\xspace}
\newcommand{\PabnH}{\ensuremath{P(\alpha,\beta)}\xspace}
\newcommand{\PwnH}{\ensuremath{P(\w)}\xspace}
\newcommand{\PDnH}{\ensuremath{P(\D)}\xspace}
\newcommand{\PwIanH}{\ensuremath{P(\w|\alpha)}\xspace}
\newcommand{\PDIwbnH}{\ensuremath{P(\D|\w,\beta)}\xspace}
\newcommand{\PDwabnH}{\ensuremath{P(\D,\w,\Ba,\beta)}\xspace}
\newcommand{\PDIwnH}{\ensuremath{P(\D|\w)}\xspace}
\newcommand{\PwIDnH}{\ensuremath{P(\w|\D)}\xspace}
\newcommand{\PwabIDnH}{\ensuremath{P(\w,\alpha,\beta|\D)}\xspace}

\newcommand{\PDwBab}{\ensuremath{P(\D,\w,\Ba,\beta|\mathcal{H})}\xspace}
\newcommand{\PwIBa}{\ensuremath{P(\w|\Ba,\mathcal{H})}\xspace}
\newcommand{\PBab}{\ensuremath{P(\Ba,\beta|\mathcal{H})}\xspace}
\newcommand{\PwBabID}{\ensuremath{P(\w,\Ba,\beta|\D,\mathcal{H})}\xspace}

\newcommand{\PBanH}{\ensuremath{P(\Ba)}\xspace}
\newcommand{\PwIBanH}{\ensuremath{P(\w|\Ba)}\xspace}

%% Snakes
\newcommand{\Esnake}{\ensuremath{\mathit{E_{snake}}}\xspace}
\newcommand{\Eimage}{\ensuremath{\mathit{E_{image}}}\xspace}
\newcommand{\Econt}{\ensuremath{\mathit{E_{cont}}}\xspace}
\newcommand{\Ecurv}{\ensuremath{\mathit{E_{curv}}}\xspace}
\newcommand{\Eint}{\ensuremath{\mathit{E_{int}}}\xspace}
\newcommand{\Eext}{\ensuremath{\mathit{E_{ext}}}\xspace}
\newcommand{\Eterm}{\ensuremath{\mathit{E_{term}}}\xspace}
\newcommand{\Eline}{\ensuremath{\mathit{E_{line}}}\xspace}
\newcommand{\Eedge}{\ensuremath{\mathit{E_{edge}}}\xspace}
\newcommand{\Econ}{\ensuremath{\mathit{E_{con}}}\xspace}
\newcommand{\Eangle}{\ensuremath{\mathit{E_{angle}}}\xspace}
\newcommand{\Elshape}{\ensuremath{\mathit{E_{lshape}}}\xspace}
\newcommand{\Eedgedir}{\ensuremath{\mathit{E_{edgedir}}}\xspace}
\newcommand{\Emodel}{\ensuremath{\mathit{E_{model}}}\xspace}
\newcommand{\wte}{\ensuremath{\mathit{w_{term}}}\xspace}
\newcommand{\wli}{\ensuremath{\mathit{w_{line}}}\xspace}
\newcommand{\wed}{\ensuremath{\mathit{w_{edge}}}\xspace}
\newcommand{\wco}{\ensuremath{\mathit{w_{con}}}\xspace}

%% Environments
\newcounter{alg}
\newenvironment{algorithm}[1]
{
    \stepcounter{alg}
    \begin{table}[htb]
    \centering
    \begin{tabular}[t]{ll}
    \hline&\\
    \multicolumn{2}{l}{\bf Algorithm \arabic{alg}: #1}\\&\\
} {
    &\\
    \hline
    \end{tabular}
    \end{table}
}
      % Include your abbreviations
\pdfminorversion=7

%% ----------------------------------------------------------------
%% --------------------THESIS/DOC INFORMATION ---------------------
\department  {School of Electronics and Computer Science}
\DEPARTMENT  {\MakeUppercase{\deptname}}
\group       {}
\GROUP       {\MakeUppercase{\groupname}}
\faculty     {Faculty of Physical Engineering and Science}
\FACULTY     {\MakeUppercase{\facname}}
\title       {Auctions for online elastic resource allocation in cloud computing}
\authors     {\texorpdfstring
             {\href{mailto:mt5g17@soton.ac.uk}{Mark Towers}}
             {Mark Towers}}
\addresses  {\groupname\\\deptname\\\univname}
\date       {\today}
\supervisor {Dr Tim Norman}
\examiner   {}
\degree     {MEng Electronic Engineering}
%% Optional Fields TODO: Replace these fields with your own data
\qualifications{}
\subject    {}
\keywords   {Auctions, reinforcement learning, cloud computing}

\begin{document}
%% ------------------ FRONT MATTER ORGANISATION -------------------
\frontmatter
\maketitle
\begin{abstract}
Edge clouds enable computational tasks to be completed at the edge of the network, without relying on access to remote
data centres. A key challenge in these settings is that limited computational resources often need to be allocated to
many self-interested users. Here, existing resource allocation approaches usually assume that tasks have inelastic
resource requirements (i.e., a fixed amount of compute time, bandwidth and storage), which may result in inefficient
resource use. In this paper, we expand previous work to an online setting such that job will arrive over time with the
task prices and resource allocation determined through training an agent using reinforcement learning.
\end{abstract}
\pdfbookmark[0]{\contentsname}{toc}
\tableofcontents
% \listoffigures
% \listoftables
%% The List of listings does not, by default, appear in the ToC, so....
\addtotoc{Listings}
% \lstlistoflistings
% \listofaddmaterial
% \addtolom{Material Name e.g Map}
% \addtolom{Material Name e.g CD}
% \addtolom{Test Material}
%% ---------- AUTHORSHIP DECLARATION/ ACKNOW. / DEDICATORY ----------
%% Either include citations like below (as many as required spaced with commas or 'and').
%% \bibentry command must be used here with prepublished papers
\authorshipdeclaration{\bibentry{Gunn:2001:pdflatex}\newline\bibentry{Lovell:2011:updated}\newline\bibentry{Gunn:2011:updated2}}
%% Or state no citations like below
%% \authorshipdeclaration{}
%% -----------------------
\acknowledgements{
This project wouldn't have started without Dr Sebastian Stein and a team of Pennsylvania State University that has
produced a paper investigating the static case of this problem. So I am grateful for the support they gave in kick
starting this project.\\ Also to my housemate for surviving with me pestering them about proof reading my paper and
this project all of the time.}

% \dedicatory{To \dots}
%%Lightweight Definitions and Abbreviations see package:nomencl for alternative
%% Include if relevant to discipline
% \listofsymbols{ll}{$w$ & The weight vector\\$\S$ & If relevant to discipline}

% Todo citep insteads of cite for adding brackets around the citation
\mainmatter
%% ------------------ MAIN MATTER (CONTENT) --------------------
\chapter{Introduction}\label{ch:introduction}
When computer programs are too large, difficult or time consuming to be run on a normal computer, these programs are
often offloaded to cloud providers like Google Cloud Platform, Amazon Web Service, Microsoft Azure and many more.
These providers all allow customers to individually request a set of resources to compute their program with. However,
this can create a bottleneck on certain resource preventing other jobs from running with this fixed requirement model.
This project considers the case where users don't request a set amount of resources but rather the user details the
total requirements for the program and a deadline for when the program must be finished. This then means that the cloud
provider can effectively balances resource demands as it has complete knowledge of its different user's requirements,
allowing more jobs to run simultaneously and lower price as there can be a lower overall demand of individual resources.

In the last few years, cloud computing~\cite{cloud_cite} has become a popular solution to run data-intensive
applications remotely. However, in some application domains, it is not feasible to rely a remote cloud, for example when
running highly delay-sensitive and computationally-intensive tasks, or when connectivity to the cloud is intermittent.
To deal with such domains, \emph{mobile edge computing}~\cite{mobile_edge_survey} has emerged as a complementary
paradigm, where computational tasks are executed at the edge of mobile networks at small data-centers, known as
\emph{edge clouds}.

Mobile edge computing is a key enabling technology for the Internet-of-Things (IoT)~\cite{mobile_edge_IoT} and in
particular applications in smart cities~\cite{mobile_edge_smart} and disaster response
scenarios~\cite{mobile_edge_disaster}. In these applications, low-powered devices generate computational tasks and
data that have to be processed quickly on local edge cloud servers. More specifically, in smart cities, these devices
could be smart intersections that collect data from road-side sensors and vehicles to produce an efficient traffic
light sequence to minimize waiting times~\cite{smart_cities_traffic_lights}; or it could be CCTV cameras that analyse
video feeds for suspicious behaviour, e.g., to detect a stabbing or other crime in progress~\cite{Sreenu2019}. In
disaster response, sensor data from autonomous vehicles (including video, sonar and LIDAR) can be aggregated in real
time to produce maps of a devastated area, search for potential victims and help first responders in focusing their
efforts to save lives~\cite{smart_disaster_management}.

To accomplish these tasks, there are typically several types of resources that are needed, including communication
bandwidth, computational power and data storage resources~\cite{vaji_infocom}, and tasks are generally delay-sensitive,
i.e., have a specific completion deadline. When accomplished, different tasks carry different values for their owners
(e.g., the users of IoT devices or other stakeholders such as the police or traffic authority). This value will depend
on the importance of the task, e.g., analysing current levels of air pollution may be less important than preventing
a large-scale traffic jam at peak times or tracking a terrorist on the run. Given that edge clouds are often highly
constrained in their resources~\cite{edge_limitations}, we are interested in allocating tasks to edge cloud servers
to maximize the overall social welfare achieved (i.e., the sum of all task values). This is particularly challenging,
because users in edge clouds are typically self-interested and may behave strategically~\cite{Bi2019} or may prefer
not to reveal private information about their values to a central allocation mechanism~\cite{Pai2013}.

An important shortcoming of existing work looking at resource allocation in edge clouds, e.g.,~\cite{vaji_infocom,
Bi2019}, is that it assumes tasks have strict resource requirements --- that is, each task consumes a
fixed amount of computation (CPU cycles per time), takes up a fixed amount of bandwidth to transfer data and uses up a
fixed amount of storage on the server. However, in practice, edge cloud servers have some flexibility in how they
allocate limited resources to each task. In more detail, to execute a task, the corresponding data and/or code first
has to be transferred to the server it is assigned to, requiring some bandwidth. This then takes up storage on the
server. Next, the task needs computing power from the server in terms of CPU cycles per time. Once computation is
complete, the results have to be transferred back to the user, requiring further bandwidth. Now, while the the storage
capacity at the server for every task is \emph{strict}, since the task cannot be run unless all the data are stored,
the bandwidth allocation and the speed at which the task is run on the server are \emph{elastic}. The latter two depend
on how tight the task's deadline is, and can be adjusted  accordingly, so that more tasks can receive service
simultaneously. Allocating the elastic resources optimally is the focus of this paper where job may be arriving over
time such for that server can redistribute their resource as new resources arrive.

\chapter{Related Works}\label{ch:related-works}
There is a considerable amount of research in the area of resource allocation and pricing in cloud computing, some of
which use auction mechanisms to deal with competition~\cite{KUMAR2017234,Zhang2017,Du2019,Bi2019}.

\section{Related work in Cloud computing}\label{sec:related-work-cloud-computing}
However, a majority of these approaches assume that users request a fixed amount of resources system resources and
processing rates, with the cloud provider having no control over the speeds, only the servers that the task was
allocated to.  In our work and appendix~\ref{sec:aamas_paper}, tasks' owners report deadlines and overall data and
computation requirements, allowing the edge cloud server to distribute its resources more efficiently based on each
task's requirements. But the appendix~\ref{sec:aamas_paper} considers the case where all jobs arrive at the same
time whereas this project considers the case of tasks arriving overtime.

Other closely related work on resource allocation in edge clouds~\cite{vaji_infocom} considers both the placement of
code/data needed to run a specific task, as well as the scheduling of tasks to different edge clouds. The goal there
is to maximize the expected rate of successfully accomplished tasks over time. Our work is different both in the setup
and the objective function. Our objective is to maximize the value over all tasks. In terms of the setup, they assume
that data/code can be shared and they do not consider the elasticity of resources.

\section{Related work in Reinforcement learning}\label{sec:related-work-reinforcement-learning}
Reinforcement learning has a huge amount of research~\cite{Sutton1998} allowing for an agent to be trained by engaging
with an environment that receives reward from taking certain actions. As our environment has action that affect the
environment and future actions, supervised and unsupervised learning are not able to be used to train the agent.
This has primarily been used in games~\cite{atari, silver2017mastering} as it allows for agent to interact with the
environment with an affect the environment which results in a reward. As our problem case allows the agent to interact
with the environment that results in rewards from the actions it takes meaning that the agent can discover the best
strategies for bidding and allocating resources instead of using heuristics predetermined by a human.

\chapter{Proposed solution}\label{ch:proposed_solution}
As the problem case has two stage that must be consider: the bidding of available jobs and the allocation of resource
for allocated jobs. Due to each of these stages requiring and returning different variables, two different function
approximation will be found, discussed in sections~\ref{sec:auction_solution} and~\ref{sec:resource_allocation}
respectively.

\begin{figure}
    \centering
    \includegraphics{extra/system_model.pdf}
    \caption{System model}
    \label{fig:system_model}
\end{figure}

A sketch of the system is shown in Fig.~\ref{fig:system_model}. 
We assume that in the system there is a set of servers $I = \{1,2,\ldots,\left|I\right|\}$ servers, which could be edge
clouds that can be accessed either through cellular base stations or WiFi access points (APs). Servers have different
types of resources: storage for the code/data needed to run a task (e.g., measured in GB), computation capacity in
terms of CPU cycles per time interval (e.g., measured in FLOP/s), and communication bandwidth to receive the data and
to send back the results of the task after execution (e.g., measured in Mbit/s). We assume that the servers are
heterogeneous in all their characteristics. More formally, we denote the storage capacity of server $i$ with $S_i$,
computation capacity with $W_i$, and the communication capacity with $R_i$.

There is a set $J = \{1,2,\ldots,\left| J \right|\}$ of  different tasks that require service from one of the servers
in set $I = \{1,2,\ldots, \left| I \right|\}$. To run any of these tasks on a server requires storing the appropriate
code/data on the same server. These could be, for example, a set of images, videos or CNN layers in identification
tasks. The storage size of task $j$ is denoted as $s_j$ with the rate at which the program is transferred to the server
$i$ at time $t$ being $s^{'}_{i,j,t}$. For a task to be computed successfully, it must fetch and execute instructions
on a CPU. We consider the total number of CPU cycles required for the program to be $w_j$, where the rate at which the
CPU cycles are assigned to the task on server $i$ at time $t$ is $w^{'}_{i,j,t}$. Finally, after the task is run and
the results obtained, the latter need to be sent back to the user. The size of the results for task $j$ is denoted with
$r_j$, and the rate at which they are sent back to the user is $r^{'}_{i,j,t}$ on server $i$ at time $t$. Every task
has a beginning time, denoted by $b_j$ and a deadline, denoted by $d_j$. This is the maximum time for the task to be
completed in order for the user to derive its value. This time includes: the time required to send the data/code to the
server, run it on the server, and get back the results. Therefore for the task to be successfully completed, it must
completed fulfill the constraint in equation~\eqref{eq:deadline}. These operations must occur in order as a server
couldn't compute a task that was not fully loaded on the machine therefore the whole of the loading stage must be
completed before moving onto the compute stage and likewise for the sending of results.

\begin{align}
    \frac{s_j}{\sum^{d_j}_{t=b_j} s^{'}_{i,j,t}} + \frac{w_j}{\sum^{d_j}_{t=b_j} w^{'}_{i,j,t}}  +
    \frac{r_j}{\sum^{d_j}_{t=b_j} r^{'}_{i,j,t}} \leq d_j && \forall{j \in J}  \label{eq:deadline}
\end{align}

As server have limited capacity, the total resource usages for all tasks running on a server must be capped to the
resource capacity. The storage constraint (equation~\eqref{eq:server_storage_capacity}) is unique as the previous amount
loaded in kept till the end of a program on server. While the computation capacity
(equation~\eqref{eq:server_computation_capacity} is the sum of compute used by all of the tasks on a server and the
bandwidth capacity (equation~\eqref{eq:server_bandwidth_capacity}) is the sum of loading and sending usages by tasks.
\begin{align}
    \sum_{j \in J} \left(\sum^{d_j}_{t=b_j} s^{'}_{i,j,t} \right) \leq S_i, && \forall{i \in I} \label{eq:server_storage_capacity} \\
    \sum_{j \in J} w^{'}_{i,j,t} \leq W_i, && \forall{i \in I, t \in T} \label{eq:server_computation_capacity} \\
    \sum_{j \in J} s^{'}_{i,j,t} + r^{'}_{i,j,t} \leq R_i, && \forall{i \in I, t \in T} \label{eq:server_bandwidth_capacity} \\
\end{align}

\section{Auction solution}\label{sec:auction_solution}
If an agent wish to run on task on the cloud, the task can be put forward with its requirements of required storage,
computation, results data and deadline. In order for fast and truthful, a reverse of the Vickrey auction~\cite{vickrey}
will be used as it is a second-price sealed-bid auction. Bidders all submit their bid for the task,
that is a single indivisible item being sold, where all bidders are incentive to bid their true valuation as the winner
with the lowest price actually pays the second lowest price. Making truthful bidding the dominant strategy for bidding
on a task preventing the need for agent to learn how to outbid another agent as it only needs to consider its
evaluation. As there is also only a single round of bidding compared to alternative auctions like English or Dutch
auctions allowing fast auctioning of task no matter the number of possible servers and can be done in parallel with
other jobs being allocated. % If to use a reserve price is unknown at this time

In order to calculate the price of the task for a server requires a understanding the resource requirements of the task,
the future supply and demand for tasks and the resource requirements of currently allocated tasks. Due to the complexity
in creating a heuristic that can accurately use this information and the amount of memory required for a table based
approach. Therefore function approximator will be used to approximate the value of the task to the server with
Long/Short Term Memory (LSTM) neural network~\cite{LSTM}. The justified for the use of this network over other models
is explained in section~\ref{sec:just_auction}.

\section{Resource allocation solution}\label{sec:resource_allocation}
In previous work (appendix~\ref{sec:aamas_paper}), that utilised a single shot problem case where jobs wouldnt arrive
over time meant that resource speed had a fixed speed and assumed that a task loading, computing and sending result
occurred concurrently. However due to the addition of time means that the task contains stages for the loading,
computing and sending of results thus requiring for the resources allocated to a task must change over time.
Therefore at each time step, a server needs to reallocate all of its resource to its currently allocated tasks as
some tasks will have completely one of its stages.

In order to select how to allocate resource to tasks, this problem doesn't seem as complex as the pricing in
section~\ref{sec:auction_solution} therefore simple heuristic and long/short term memory neural network will be
implemented. This is justified in section~\ref{sec:just_resource_allocation}.


% Justiifcation of the network type
% Justification of the learning method (reinforcement, unsupervised, supervised)
\begin{table}[]
    \centering
    \begin{tabular}{|l|c|c|}
        Neural Network & Advantages & Disadvantages \\ \hline
        Artificial neural networks (ANN)
        Long/Short Term Memory (LSTM) & & \\ \hline
        Recurrent neural network (RNN) & & \\ \hline
        Gated Recurrent unit (GRU) & & \\ \hline
        Convolutional neural network (CNN) & & \\ \hline
        Differentiable neural computer (DNC) & & \\ \hline
        Neural Turing Machine (NTM) & & \\ \hline
    \end{tabular}
    \caption{Neural network zoo}
    \label{tab:neural_networks}
\end{table}

\chapter{Justification of the approach}\label{ch:justification-of-the-approach}
The proposed solution in section~\ref{ch:proposed_solution} has been chosen due to a variety of reasons.

\section{Justification for the auction} \label{sec:just_auction}
The vickrey auction (~\cite{vickrey}) has been chosen as it a dominant strategy for agents is to truthfully report
the their true evaluation of a task to it. This project will require to reverse as it is the server are selling their
resources and buying the tasks for a price with the lowest price winning instead of the standard pricing model where
the  highest price winning the item. However dispute this change of swapping the pricing model, the dominant strategy
is truthful reporting as if server report negative prices for a task then the highest negative number is equal to the
lowest positive number.

\section{Justification for resource allocation} \label{sec:just_resource_allocation}
Because it is the best idea


\chapter{Work requirements}\label{ch:work-requirements}
An estimate of any support required to complete the project
* Only Iridis 4 with GPU (currently have access to Iridis 4)

\section{Work to date}\label{sec:work-to-date}
As this project is an extension to previous work done in the Agent, Interaction and Complexity research labs that has
produced the paper in Section~\ref{sec:aamas_paper}. The majority of this research occurred over the summer of 2019
with the paper that is currently under peer review done from October 2019 to 15th November 2019. The paper produced
was done with support from Dr Fidan Mehmeti and Dr Sebastian Stein with myself being the primary author.

For the remaining time, I have studied reinforcement learning that is the primary technological additional that will
be used in the proposed solution (section~\ref{ch:proposed_solution}) and described in
Section~\ref{sec:related-work-reinforcement-learning}.

\begin{figure}
    \centering
    \includegraphics[width=\linewidth]{past_work_grantt/past_grantt.pdf}
    \caption{Work that has been done to date}
    \label{fig:past_grantt}
\end{figure}

\section{Plan of the remaining work}
Due to this term having been completing the paper (Section~\ref{sec:aamas_paper}), I have not done any programming
towards the project. Therefore the begin of the next term will be build a framework for which different pricing and
resource allocation heuristics can be applied and compared. Once this has been completed, analysis and comparison of
the heuristic will be done with different server and task models.
\begin{figure}
    \centering
    \includegraphics[width=\linewidth]{future_work_grantt/future_grantt}
    \caption{}
    \label{fig:future_grantt}
\end{figure}

\backmatter

\appendix
\section*{}\label{sec:aamas_paper} % Insert the AAMAS paper
\includepdf[pages={1}, offset=-25mm -20mm]{extra/aamas_2020}
\includepdf[pages={2}, offset= 25mm -20mm]{extra/aamas_2020}
\includepdf[pages={3}, offset=-25mm -20mm]{extra/aamas_2020}
\includepdf[pages={4}, offset= 25mm -20mm]{extra/aamas_2020}
\includepdf[pages={5}, offset=-25mm -20mm]{extra/aamas_2020}
\includepdf[pages={6}, offset= 25mm -20mm]{extra/aamas_2020}
\includepdf[pages={7}, offset=-25mm -20mm]{extra/aamas_2020}
\includepdf[pages={8}, offset= 25mm -20mm]{extra/aamas_2020}

\bibliographystyle{plainnat}
\bibliography{references}
\end{document}
%% ----------------------------------------------------------------
